\documentclass{letter}

\usepackage{amssymb}
\usepackage{amsmath}

\begin{document}

Basics: \\

$P(n)=a_1 x^n + a_2 x^{n-1} + \cdots + a_{n-1}x +  a_{n}$ \\
$P(m)=a_1 x^m + a_2 x^{m-1} + \cdots + a_{m-1}x +  a_{m}$ \\

$n < m: 
\lim_{x\to \infty} \frac{P(n)}{P(m)} = 0 $\\

$n = m: 
\lim_{x\to \infty} \frac{P(n)}{P(m)} = Const $\\

$n > m: 
\lim_{x\to \infty} \frac{P(n)}{P(m)} = \infty $\\


Patterns: \\
1. $\lim_{x\to x_0} \sqrt{a} - \sqrt{b} = 
\frac{(\sqrt{a} - \sqrt{b})(\sqrt{a} + \sqrt{b})}{(\sqrt{a} + \sqrt{b})} =
\frac{a - b}{(\sqrt{a} + \sqrt{b})}  = \cdots $

2. $ \lim_{x\to x_0} \log_a{f(x)} =  \log_a {\lim_{x\to x_0} {f(x)} } =  $

3. $ \lim_{x\to x_0} e^{f(x)} =  e^ {\lim_{x\to x_0} {f(x)} } =  $

Remarkable limits: \\
1. $ \lim_{x\to 0} \frac{\sin{x}}{x} = 0 (\lim_{x\to 0} \frac{\tan{x}}{x} = 0) $ \\
2. $ \lim_{x\to 0} (1 + 1/x)^x = e $ \\
3. $ \lim_{x\to 0} \frac{\ln{(1+x)}}{x} = 1 $ \\
4. $ \lim_{x\to 0} \frac{a^x + 1}{x} = \ln{a} $ \\
5. $ \lim_{x\to 0} \frac{(1 + x)^a - 1}{x} = a $ \\


Landau symbols: \\
$ f(x) = o( \phi(x) ) \Leftrightarrow \lim_{x\to x_0} \frac{f(x)}{\phi(x) = 0} $ \\
e.g. $\phi(x) >> f(x)$ in $U_x$

$ \alpha < \beta $ \\
$x^{\alpha} = o(x^{\beta}, x\to \infty ) $ \\
$x^{\beta} = o(x^{\alpha}, x\to 0 )$ \\

Examples: \\
$ f(x) = o( 1 ) $ \\
$ x = o( x^2 ), x\to \infty $ \\
$ x^2 = o( x ), x\to 0 $ \\

Teylor formula:
$$
f(x) = \sum_{k = 0}^{n - 1} \frac{f^{(k)} (a)}{k!} (x - a)^k + R_n(x), x \in (a, b)
$$
$$
R_n(x) = \frac{f^{(n)}(a + \Theta(x - a))}{n!} 
$$

Teylor series: \\
1. $e^x = 1 + x + \frac{x^2}{x!} + \frac{x^3}{3!} + \cdots + \frac{x^n}{n!} + o(x^n) $ \\
2. $ \sin{x} = x - \frac{x^3}{3!} + \frac{x^5}{5!} + \cdots + (-1)^{n-1}\frac{x^{2n-1}}{(2n-1)!} + o(x^{2n}) $ \\
3. $ x = x - \frac{x^2}{2!} + \frac{x^4}{4!} + \cdots + (-1)^n\frac{x^{2n}}{(2n)!} + o(x^{2n+1}) $ \\
4. $ (1+x)^n = 1 + mx + x^2\frac{m(m-1)}{2!} + x^n\frac{m(m-1)\cdots(m-n+1)}{n!} + o(x^n), x \in (-1,1) $ \\
5. $ \ln(x + 1) = x - \frac{x^2}{2} + \frac{x^3}{3} - \cdots + (-1)^{n-1}\frac{x^n}{n} + o(x^n), x \in (-1,1) $ \\


\end{document}