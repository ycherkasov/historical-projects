\section{Тригонометрия}


\subsection{Основные формулы}

$\sin^2 \alpha + \cos^2 \alpha = 1 ,  \forall \alpha $ \\
$\tg^2 \alpha + 1 = \frac{1}{\cos^2 \alpha} = \sec^2 \alpha $, $\alpha \neq \frac{\pi}{2} + \pi n, n \in \mathbb Z $ \\
$\ctg^2 \alpha + 1 = \frac{1}{\sin^2 \alpha} = \cosec^2 \alpha$ \\

\subsection{Формулы приведения}

Формулами приведения называются формулы следующего вида:

$ f ( n \pi + \alpha )  = \pm  f (\alpha)$

$ f ( n \pi - \alpha )  = \pm  f (\alpha)$

$ f \left(  \frac{(2n+1) \pi}{2} + \alpha\right)  = \pm  g (\alpha)$

$ f \left(  \frac{(2n+1) \pi}{2} - \alpha\right)  = \pm  g (\alpha)$

Здесь $f$ — любая тригонометрическая функция, $ g $ — соответствующая ей кофункция (то есть косинус для синуса, синус для косинуса, тангенс для котангенса, котангенс для тангенса, секанс для косеканса и косеканс для секанса), $ n $ — целое число. Перед полученной функцией ставится тот знак, который имеет исходная функция в заданной координатной четверти при условии, что угол $ \alpha $ острый, например:

$ \cos \left(  \frac{ \pi}{2} - \alpha \right)  =   \sin \alpha$

\subsection{Сумма и разность углов}

$\sin \left( \alpha \pm \beta \right) = \sin \alpha \cos \beta \pm \cos \alpha \sin \beta$ \\
$\cos \left( \alpha \pm \beta \right) = \cos \alpha \cos \beta \mp \sin \alpha \sin \beta$ \\
$\tg \left( \alpha \pm \beta \right) = \frac{ \tg \alpha \pm \tg \beta}{1 \mp \tg \alpha \tg\beta}$ \\
$\ctg \left( \alpha \pm \beta \right) = \frac{ \ctg \alpha \ctg \beta \mp 1}{\ctg \beta \pm \ctg\alpha}$ \\

\subsection{Формулы двойного, тройного и половинного углов}

$ \sin 2 \alpha = 2 {\sin \alpha}{\cos \alpha} $ \\
$ \cos 2 \alpha = {\cos^2 \alpha} - {\sin^2 \alpha} = 2 {\cos^2 \alpha} - 1 = 1 - 2 {\sin^2 \alpha} $ \\
$ \tg 2 \alpha = \frac{2 \tg \alpha}{1 - \tg^2 \alpha} $ \\
$ \ctg 2 \alpha = \frac{\ctg^2 \alpha - 1}{2 \ctg \alpha} $ \\

$\sin 3\alpha = 3 \sin \alpha - 4 \sin^3\alpha \,$ \\
$\cos 3\alpha = 4 \cos^3\alpha - 3 \cos \alpha \,$ \\
$\tg 3\alpha = \frac{3 \tg\alpha - \tg^3\alpha}{1 - 3 \tg^2\alpha}$ \\
$\ctg 3\alpha = \frac{3 \ctg\alpha - \ctg^3\alpha}{1 - 3 \ctg^2\alpha}$ \\

\subsection{Формулы понижения степени}

$\sin^2\alpha = \frac{1 - \cos 2\alpha}{2}$ \\
$\cos^2\alpha = \frac{1 + \cos 2\alpha}{2}$ \\
$\sin^2\alpha \cos^2\alpha = \frac{1 - \cos 4\alpha}{8}$ \\
$\sin^3\alpha = \frac{3 \sin\alpha - \sin 3\alpha}{4}$ \\
$\cos^3\alpha = \frac{3 \cos\alpha + \cos 3\alpha}{4}$ \\
$\sin^3\alpha \cos^3\alpha = \frac{3\sin 2\alpha - \sin 6\alpha}{32}$ \\
$\sin^4\alpha = \frac{3 - 4 \cos 2\alpha + \cos 4\alpha}{8}$ \\
$\cos^4\alpha = \frac{3 + 4 \cos 2\alpha + \cos 4\alpha}{8}$ \\
$\sin^4\alpha \cos^4\alpha = \frac{3-4\cos 4\alpha + \cos 8\alpha}{128}$ \\
$\sin^5\alpha = \frac{10 \sin\alpha - 5 \sin 3\alpha + \sin 5\alpha}{16}$ \\
$\cos^5\alpha = \frac{10 \cos\alpha + 5 \cos 3\alpha + \cos 5\alpha}{16}$ \\
$\sin^5\alpha \cos^5\alpha = \frac{10\sin 2\alpha - 5\sin 6\alpha + \sin 10\alpha}{512}$ \\

\subsection{Формулы приведения суммы к произведению и наоборот}

$ \sin  \alpha  \sin  \beta = \frac{ \cos ( \alpha - \beta) -  \cos ( \alpha + \beta)}{2} $ \\
$ \sin  \alpha  \cos  \beta = \frac{ \sin ( \alpha + \beta) +  \sin ( \alpha - \beta)}{2} $ \\
$ \cos  \alpha  \cos  \beta = \frac{ \cos ( \alpha + \beta) +  \cos ( \alpha - \beta)}{2} $ \\

$ \sin  \alpha \pm  \sin  \beta = 2 \sin \frac{ \alpha \pm \beta}{2} \cos \frac{ \alpha \mp \beta}{2}$ \\
$ \cos  \alpha + \cos  \beta = 2 \cos \frac{ \alpha + \beta}{2} \cos \frac{ \alpha - \beta}{2}$ \\
$ \cos  \alpha - \cos  \beta = - 2 \sin \frac{ \alpha + \beta}{2} \sin \frac{ \alpha - \beta}{2}$ \\
$ \tg  \alpha \pm \tg  \beta = \frac{ \sin ( \alpha \pm \beta)}{ \cos  \alpha \cos  \beta}$ \\
$ \ctg  \alpha \pm \ctg  \beta = \frac{ \sin ( \beta \pm \alpha)}{ \sin  \alpha \sin  \beta}$ \\

\subsection{Формулы Эйлера}

$~e^{ix}=\cos x+i\sin x$ \\
$\sin x=\frac{e^{ix}-e^{-ix}}{2i}$ \\
$\cos x=\frac{e^{ix}+e^{-ix}}{2}$ \\
$\tg x = \frac{i(e^{-ix}-e^{ix})}{e^{ix}+e^{-ix}}$ \\
$\ctg x = \frac{i(e^{ix}+e^{-ix})}{e^{ix}-e^{-ix}}$ \\
