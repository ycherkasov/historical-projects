\section{Правила и формулы дифференцирования. Геометрический и физический смысл производной}

\subsection{Логарифмическое дифференцирование}

\subsubsection{Пример 1}

DВычислить производную

$ y = \sqrt[3]{ \dfrac{x^{2}( x+1 )}{ x-3 } } $

$ y' = y \ln' y $

$ \ln y = \dfrac{1}{3} 
\left(
2 \ln x + \ln(x+1) - \ln(x-3)
\right)  $

$ \ln' y = \dfrac{1}{3} 
\left(
\dfrac{2}{x} + \dfrac{1}{x+1} - \dfrac{1}{x-3}
\right)  $

Приводим подобные

$ y' = \sqrt[3]{ \dfrac{x^{2}( x+1 )}{ x-3 } } $
$ \left( \dfrac{2x^{2} - 8x - 6}{3x(x+1)(x-3)} \right)  $

\subsubsection{Пример 2}

$ y = \tg x^{\ctg x}$

$ y' = y \ln' y $

$  \ln y = \ctg x \ln \tg x $

$  \ln' y = \ctg' x \ln \tg x + \ctg x (\ln \tg x)' $
$ = -\dfrac{\ln \tg x}{\sin^{2} x} +\dfrac{ \ctg x}{\tg x} \dfrac{1}{\cos^{2} x} $
$ = \dfrac{ \cos^{2} x}{\sin^{2} x} \dfrac{1}{\cos^{2} x} - \dfrac{\ln \tg x}{\sin^{2} x} $
$ = \dfrac{1 - \ln \tg x}{\sin^{2} x} $

$ y' = \tg x^{\ctg x} \left( \dfrac{1 - \ln \tg x}{\sin^{2} x} \right)  $

\subsection{Геометрический смысл производной}

\subsubsection{Пример 1}

Пусть задана функция $ f(x) = e^{x} $

Длину отрезка от касательной в точке $ x = 0 $ до пересечения её с осью $ Ox $

Расстояние между двумя точкамы вычисляется по формуле

$ d(A,B) = \sqrt{(x_{1} - x_{0})^{2} + (y_{1} - y_{0})^{2}} $

Очевидно, точка касания $ x_0 = 0, y_0 = f(0) = 1, y_1 = 0 $ (т.к. перечечение с осью $ Ox $)

Найдем уравнение касательной

$ y - f(x_0) = f'(x_0)(x - x_0) $

$ y - 1 = f'(x_0)x, f'(0) = 1 $

$ y - 1 = x, f'(0) = 1, y_0 = 0 $

$ x_1 = -1 $

$ d(A,B) = \sqrt{(-1)^{2} + (-1)^{2}} = \sqrt{2}$

\subsubsection{Пример 2}

Найти угол пересечения $ \alpha $ кривых

$ y = x^{2} $ и $ x = y^{2} $

Формула угла пересечения кривых

$$
\tg \alpha = \dfrac{k_{1} - k_{2}}{1 - k_{1}k_{2}} \label{functions_angle}
$$

где $ k_1, k_2 $ - угловые коэффициенты касательных

$ k_1 = (x^{2})' = 2x $

$ k_2 = (\sqrt{x})' = \dfrac{1}{2 \sqrt{x}} $

Теперь найдем точки пересечения


$\begin{cases}
 y = x^{2} \\
 x = y^{2}
\end{cases}$

$\begin{cases}
 y - x^{2} = 0 \\
 x = y^{2}
\end{cases}$

$ y - y^{4} = 0 $

$ y(1 - y^{3}) = 0 $

$ x_1 = 1, y_1 = 1 $

$ x_2 = 0, y_2 = 0 $

TODO - funtion graph!!!

$ \tg \alpha_1 = \dfrac{k_{1} - k_{2}}{1 - k_{1}k_{2}} = \dfrac{3}{4}, \alpha_1 \approx 37^{\circ} $

$ \tg \alpha_2: k_1 = 0, k_2 = \infty \Rightarrow \alpha_2 = \dfrac{\pi}{2}  $

\subsubsection{Пример 3}

Висячий мост имеет форму параболы, причем длина между концами 80м, а стрелка - 5 м.

Найти угол провисания $ \alpha $

Очевидно, уравнение параболы

$ y = a x^{2} $

$ 5 = a 40^{2} $

$ a = \dfrac{5}{1600} = \dfrac{1}{320} $

$ y = \dfrac{x^{2}}{320} $

$ y' = \dfrac{x}{160} $



Уравнение касательной в точке $ (-40) $

$ y - f(x_0) = f'(x_0)(x - x_0) $

$ y - 5 = \dfrac{5}{160}(x + 40) $ 

$ y - 5 = \dfrac{5}{160}x + 6,25 $ 

$ k = \dfrac{5}{160} = tg \alpha $

$ \alpha = \arctg \dfrac{5}{160} $

\subsection{Физический смысл производной}

\subsubsection{Пример 1}

Пусть стороны прямоугольнока растут по следующему закону

$\begin{cases}
b = 3t + 1 \\
c = 2t + 5
\end{cases}$

Найти закон, по которому растут площадь $ S(t) $ и периметр $ P(t) $, 
найти скорость роста в моент времени $ t = 5 $

$ S(t) = (3t+1)(2t+5) = 6t^{2} + 17t + 5 $

$ P(t) = 2(3t+1+2t+5) = 10t + 12 $

$ V_{S}(t) = (6t^{2} + 17t + 5)' = 12t + 17 $

$ V_{S}(5) = 77 $

$ V_{P}(t) = 10 $, скорость постоянна

\subsubsection{Пример 2}

Закон радиоактивного распада вещества

$ m = m_0 e^{-kt}, k $ -- постоянная распада

Найти закон скорости и ускорения распада

$ v = (m_0 e^{-kt})' = -k m_0 e^{-kt} $

$ a = (-k m_0 e^{-kt})' = k^{2} m_0 e^{-kt} $