\section{Сложные проценты}

\subsection{Базовая формула}

Если сложные проценты начисляются раз в год, начальная сумма $ PV $ (present value), ожидаемая $ FV $ (future value), процентная ставка $ i $, то сумма за $ n $ лет будет

$$
FV = PV(1 + i)^{n}
$$

Если сложные проценты начисляются $ m $ раз в год, то эффективная процентная $ EFR $ ставка будет

$$ 
EFR = (1 + \dfrac{i}{m}) - 1
$$

Тогда $ FV = PV(1 + EFR)^{n} $

\subsection{Приведенная стоимость}

Когда мы хотим узнать, какую сумму $ PV $ (приведенная стоимость) нужно вложить, чтобы при ставке $ i $ получить $ FV $

$$
PV = \dfrac{1}{(1+i)^{n}}
$$

\subsection{Дисконтирование и оценка инвестиций}

При оценке различных инвестиционных проектов сравнивается один из показателей

\begin{itemize}
\item 
Оценка будущей стоимости

\item 
Оценка процентной ставки

\item 
Оценка времени, в течение которого будет получена сумма $ FV $

\end{itemize}