\section{Бесконечно малые и бесконечно большие функции и их свойства. Арифметические свойства пределов}

Бесконечно малая (величина) — числовая функция или последовательность, которая стремится к нулю.

Бесконечно большая (величина) — числовая функция или последовательность, которая стремится к бесконечности определённого знака.

Исчисление бесконечно малых — вычисления, производимые с бесконечно малыми величинами, при которых производный результат рассматривается как бесконечная сумма бесконечно малых. Исчисление бесконечно малых величин является общим понятием для дифференциальных и интегральных исчислений, составляющих основу современной высшей математики.

\subsection{Свойства бесконечно малых}

\begin{itemize}
\item 
Сумма конечного числа бесконечно малых — бесконечно малая.

\item 
Произведение бесконечно малых — бесконечно малая.

\item 
Произведение бесконечно малой последовательности на ограниченную — бесконечно малая. Как следствие, произведение бесконечно малой на константу — бесконечно малая.

\item 
Если $a_n$ — бесконечно малая последовательность, сохраняющая знак, то $b_n=\frac{1}{a_n}$ — бесконечно большая последовательность.
\end{itemize}


\subsection{Теорема. Связь функции с ее пределом и бесконечно малой}

Для того, чтобы число A являлось пределом функции при $ x \to a $, н. и д., чтобы $ x $ предствлялся в виде суммы

$ f(x) = A + \alpha(x) $,+

где $ \alpha(x) $ - бмф при $ x \to a $

\subsection{Односторонние пределы}

В определении предела 

$\lim\limits_{x \to x_0} f \left( x \right) = A \Leftrightarrow \forall \varepsilon > 0 ~ \exists \delta = \delta \left( \varepsilon \right) ~ \forall x \colon 0 < \left| x - x_0 \right| < \delta \Rightarrow \left| f \left( x \right) - A \right| < \varepsilon$

заменим неравенство с модулем

$ \Leftrightarrow \forall \varepsilon > 0 ~ \exists \delta = \delta \left( \varepsilon \right) ~ \forall x \colon x_0 - \delta < x < x_0 \Rightarrow \left| f \left( x \right) - A \right| < \varepsilon $ - 

предел слева.
Для предела справа - аналогично.

Если пределы справа и слева совпадают, то функция имеет предел в этой точке.


