\section{Функция. Предел функции в точке и бесконечности. Теоремы о пределах}

\subsection{Определение}

Функция (отображение, оператор, преобразование) — математическое понятие, отражающее связь между элементами множеств. Можно сказать, что функция — это «закон», по которому каждому элементу одного множества (называемому областью определения) ставится в соответствие некоторый элемент другого множества (называемого областью значений).
\\\\
В теоретической математике функцию $f$ удобно определить как бинарное отношение (то есть множество упорядоченных пар $(x,y)\in X\times Y$), которое удовлетворяет следующему условию: для любого $x\in X$ существует единственный элемент $y\in Y$ такой, что $(x,y)\in f$.
\\\\
Это и позволяет говорить о том, что элементу $x\in X$ сопоставлен \textbf{один и только один} элемент $y\in Y$ такой, что $(x,y)\in f$.

\subsection{Предел функции}

\subsubsection{Теорема о единственности предела}
