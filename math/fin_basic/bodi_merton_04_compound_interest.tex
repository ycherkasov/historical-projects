\section{Сложные проценты}

\subsection{Базовая формула}

Если сложные проценты начисляются раз в год, начальная сумма $ PV $ (present value), ожидаемая $ FV $ (future value), процентная ставка $ i $, то сумма за $ n $ лет будет

$$
FV = PV(1 + i)^{n}
$$

Если сложные проценты начисляются $ m $ раз в год, то эффективная процентная $ EFR $ ставка будет

$$ 
EFR = (1 + \dfrac{i}{m}) - 1
$$

Тогда $ FV = PV(1 + EFR)^{n} $

\subsection{Приведенная стоимость}



Когда мы хотим узнать, какую сумму $ PV $ (приведенная стоимость) нужно вложить, чтобы при ставке $ i $ получить $ FV $

$$
PV = \dfrac{1}{(1+i)^{n}}
$$

\subsection{Дисконтирование и оценка инвестиций}

При оценке различных инвестиционных проектов сравнивается один из показателей

\begin{itemize}
\item 
Оценка будущей стоимости

\item 
Оценка процентной ставки

\item 
Оценка времени, в течение которого будет получена сумма $ FV $

\end{itemize}

\subsection{Аннуитеты}

Будущая стоимость аннуитетных платежей предполагает, что платежи осуществляются на приносящий проценты вклад. Поэтому будущая стоимость аннуитетных платежей является функцией как величины аннуитетных платежей, так и ставки процента по вкладу.

Будущая стоимость серии аннуитетных платежей (FV) вычисляется по формуле (предполагается сложный процент)

$$
FV_\mathrm{annuity} = X \dfrac{(1+r)^n-1}{r} 
$$

где $ r $ -- ставка процента, n -- количество периодов, в которые осуществляются аннуитетные платежи, $ X $ -- величина аннуитетного платежа.

Срочный аннуитет в рассматриваемом случае начисления процентов по аннуитетным платежам, имеет на один год начисления процентов больше. Поэтому формула для вычисления будущей стоимости аннуитета пренумерандо приобретает следующий вид

$$
FV_\mathrm{annuity} = X (1+r) \dfrac{(1+r)^n-1}{r}
$$