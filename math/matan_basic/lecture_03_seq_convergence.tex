\section{Сходимость числовой последовательности. Бесконечно малые и бесконечно большие поcледовательности. Число е}

\subsection{Ограниченная последовательность}

Ограниченная сверху последовательность — это последовательность элементов множества $X$, все члены которой не превышают некоторого элемента из этого множества. Этот элемент называется верхней гранью данной последовательности.

$ \Leftrightarrow \exists M \in X  \forall n \in N \colon x_n \leqslant M $

Ограниченная снизу последовательность — это последовательность элементов множества $X$, для которой в этом множестве найдётся элемент, не превышающий всех её членов. Этот элемент называется нижней гранью данной последовательности.

$\Leftrightarrow \exists m \in X  \forall n \in N \colon x_n \geqslant m $

Ограниченная последовательность (ограниченная с обеих сторон последовательность) — это последовательность, ограниченная и сверху, и снизу

$\Leftrightarrow \exists m,M \in X  \forall n \in N \colon m \leqslant x_n \leqslant M $

Неограниченная последовательность — это последовательность, которая не является ограниченной

\subsection{Бесконечно большая(малая) последовательность}

Бесконечно малая последовательность — это последовательность, предел которой равен нулю.

Бесконечно большая последовательность — это последовательность, предел которой равен (плюс-минус) бесконечности.

\subsection{Свойства последовательностей}

% список
\begin{itemize}

\item 
Алгебраическая сумма любого конечного числа бесконечно малых последовательностей сама также является бесконечно малой последовательностью.

\item 
Произведение ограниченной последовательности на бесконечно малую последовательность есть бесконечно малая последовательность.

Например, $\lim\limits_{n \to \infty} \dfrac{\cos{n^2}}{n^2} = 0$, т.к. $\cos{n^2}$ - ограниченная, $\frac{1}{n^2}$ - бесконечно малая

\item 
Произведение любого конечного числа бесконечно малых последовательностей есть бесконечно малая последовательность.

\item 
Любая бесконечно малая последовательность ограничена.

\item 
Любая бесконечно большая последовательность неограничена. Обратное неверно.

Например, $\{ n \sin{\frac{n \pi}{2}}  \} = \{ 0,-2,0,4,0,-6 \}$ - не имеет $\lim = \infty$, но неограничена
\end{itemize}

\subsection{Теорема об ограниченности сходящейся последовательности}

Всякая сходящаяся последовательность является ограниченной.

Доказательство - из геометрического смысла сходимости - поиск максимального и минимального элементов.


\subsection{Принцип вложенных отрезков}

Бесконечное число вложенных отрезков имеет единственную точку.

Применяется в доказательствах теорем.

\subsection{Число е}

TODO

\subsection{Частичный предел}

Частичным пределом последовательности называется предел какой-либо её подпоследовательности, если существует хотя бы одна подпоследовательность, имеющая предел.

Теорема Больцано-Вейершттасса

Из всякой ограниченной последовательности точек пространства $\mathbb{R}^n$ можно выделить сходящуюся подпоследовательность.

Доказательство - исходя из принципа сложенных отрезков. Делим пополам отрезок, и берем тот, который содержит $\infty$ точек, делаем так, пока не получим предел.

Следствия :

\begin{itemize}
\item 
Если все частичные пределы равны, то последовательность сходится.

\item 
Если хотя бы два частичных предела не равны, то последовательность не сходится.

\end{itemize}

\subsection{Критерий Коши (фундаментальная последовательность)}

Для любого $\varepsilon > 0$ существует такое натуральное число $N_\varepsilon$, что $\rho(x_{n}, x_{m}) < \varepsilon\ $ для всех $ n, m > N_\varepsilon$, где $\rho(x_{n}, x_{m})$ - расстояние между точками последовательности.