\section{Функция. Предел функции в точке и бесконечности. Теоремы о пределах}

\subsection{Определение}

Функция (отображение, оператор, преобразование) — математическое понятие, отражающее связь между элементами множеств. Можно сказать, что функция — это «закон», по которому каждому элементу одного множества (называемому областью определения) ставится в соответствие некоторый элемент другого множества (называемого областью значений).
\\\\
В теоретической математике функцию $f$ удобно определить как бинарное отношение (то есть множество упорядоченных пар $(x,y)\in X\times Y$), которое удовлетворяет следующему условию: для любого $x\in X$ существует единственный элемент $y\in Y$ такой, что $(x,y)\in f$.
\\\\
Это и позволяет говорить о том, что элементу $x\in X$ сопоставлен \textbf{один и только один} элемент $y\in Y$ такой, что $(x,y)\in f$.

\subsection{Предел функции}

Предел функции является обобщением понятия предела последовательности.

Наиболее часто определение предела функции формулируют на языке окрестностей. То, что предел функции рассматривается только в точках, предельных для области определения функции, означает, что в каждой окрестности данной точки есть точки области определения; это позволяет говорить о стремлении аргумента функции (к данной точке).

\subsection{Предел функции по Коши}

Значение $~A$ называется \textit{пределом} функции $f \left( x \right)$ в точке $~x_0$, если для любого наперёд взятого положительного числа $\varepsilon$ найдётся отвечающее ему положительное число $\delta = \delta \left( \varepsilon \right)$ такое, что для всех аргументов $~x$, удовлетворяющих условию $0 < \left| x - x_0 \right| < \delta$, выполняется неравенство $\left| f \left( x \right) - A \right| < \varepsilon$.

$\lim\limits_{x \to x_0} f \left( x \right) = A \Leftrightarrow \forall \varepsilon > 0 ~ \exists \delta = \delta \left( \varepsilon \right) ~ \forall x \colon 0 < \left| x - x_0 \right| < \delta \Rightarrow \left| f \left( x \right) - A \right| < \varepsilon$

\subsection{Предел функции по Гейне}

Предел по Гейне более интуитивен, т.к. определяет предел через сходимость множеств.

Значение $~A$ называется пределом функции $f \left( x \right)$ в точке $~x_0$, если для любой последовательности точек $\left\{ x_n \right\}_{n=1}^{\infty}$, сходящейся к $~x_0$, но не содержащей $~x_0$ в качестве одного из своих элементов (то есть в проколотой окрестности $~x_0$ ), последовательность значений функции $\left\{ f \left( x_n \right) \right\}_{n=1}^{\infty}$ сходится к $~A$.

$$
\lim\limits_{x \to x_0} f \left( x \right) = A \Leftrightarrow \forall \left\{ x_n \right\}_{n = 1}^{\infty} \left( \forall n \in N \colon x_n \neq x_0 \right) 
$$
$$
\lim\limits_{n \to \infty} x_n = x_0 \Rightarrow \lim_{n \to \infty} f \left( x_n \right) = A
$$

\subsubsection{Теоремы о пределах}

\begin{enumerate}
\item 
Если предел функции существует, он единственный.
Меняем кванторы в определении предела по Коши.

\item 
Если функция имеет предел, она ограничена.

\item 
Если задано две функции в окрестности точки $ A $ $ f(x) $ и $ \phi(x) $,
то их пределы связаны тем же неравенством, что и функции.

\item
"Лемма о двух милиционерах". Если функция $y=f(x)$ такая, что $\varphi(x)\leqslant f(x)\leqslant\psi(x)$ для всех $x$ в некоторой окрестности точки $a$, причем функции $\varphi(x)$ и $\psi(x)$ имеют одинаковый предел при $x\to a$, то существует предел функции $y=f(x)$ при $x\to a$, равный этому же значению, то есть
: $\lim_{x\to a}\varphi(x)=\lim_{x\to a}\psi(x)=A\Rightarrow\lim_{x\to a}f(x)=A.$
\end{enumerate}


\subsubsection{Пределы на бесконечности}

Предел функции на бесконечности описывает поведение значения данной функции, когда её аргумент становится бесконечно большим.



