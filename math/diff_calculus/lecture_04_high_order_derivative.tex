\section{Производные высших порядков. Дифференциалы высших порядков. Дифференцирование функции, заданной параметрически. Вектор-функция скалярного аргумента}

\subsection{Производные высших порядков. Дифференциалы высших порядков.}

Понятие производной произвольного порядка задаётся рекуррентно. Полагаем

$f^{(0)}(x_0) \equiv f(x_0)$

Если функция $f$ дифференцируема в $x_0$, то производная первого порядка определяется соотношением

$f^{(1)}(x_0) \equiv f'(x_0)$

Пусть теперь производная $n$-го порядка $f^{(n)}$ определена в некоторой окрестности точки $x_0$ и дифференцируема. Тогда

$f^{(n+1)}(x_0) = \left(f^{(n)}\right)'(x_0)$

Физический смысл производной второго порядка $ f''(x) $ проясняется из того, что если первая производная $ f'(x)$ задаёт мгновенную скорость изменения значений $ f(x)$ в момент времени $x$, то вторая производная, то есть производная от $ f'(x)$, задаёт мгновенную скорость изменения значений мгновенной скорости, то есть ускорение значений $ f(x)$. Следовательно, третья производная $ f'''(x) $ -- это скорость изменения ускорения (или, что то же самое, ускорение изменения скорости, поскольку, как очевидно следует из определения.

Геометрический смысл второй производной связан с понятиями выпуклости и кривизны графика функции.


Формулы производной произведения и отношения обобщаются на случай n-кратного дифференцирования (формула Лейбница):

\subsection{Формула Лейбница для произведения производных}

$(f g)^{(n)}=\sum\limits_{k=0}^{n}{C_n^k f^{(n-k)} g^{(k)}},$ где $C_n^k$ — биномиальные коэффициенты.

\subsubsection{Пример}

$ y = x^{2} e^{x} $

Найдем тысячную производную по формуле Лейбница

$ y^{(1000)} = 
(e^{x})^{(1000)} x^{2} +
\dfrac{1000}{1} (e^{x})^{(999)} (x^{2})' +
\dfrac{1000 \cdot 999}{1} (e^{x})^{(998)} (x^{2})'' + 0
$

Более чем вторую производную для $ x^{2} $ вычислить нельзя.

$ y^{(1000)} = 
e^{x} (x^{2} + 2000 x + 499500) $

\subsection{Производные высших порядков для взаимно обратных функций}

Пусть $ y = f(x), x = g(y) $, $ f,g $ -- взаимно обратные функции

Тогда $ x'_{y} = \dfrac{1}{y'_{x}} $

Тогда $ x''_{y} = 
\left(
\dfrac{1}{y'_{x}}
\right)'
=
\left(
\dfrac{1}{ \frac{dy}{dx} }
\right)'
=
\left(
\dfrac{dx}{dy}
\right)'
=
\dfrac{d}{dy} (x'_y)
$

Подставим значение из формулы обратной производной

$
\dfrac{d}{dy} (x'_y)=
\dfrac{d}{dy} 
\dfrac{1}{y'_{x}}
$

Очевидно, это производная сложной функции. По формуле дифференцировнаия сложной функции 
$ \frac{dy}{dx} = \frac{dy}{du}\frac{du}{dx} $

$
\dfrac{d}{dy} 
\dfrac{1}{y'_{x}}
=
\underbrace{
\dfrac{d}{dx}
\dfrac{1}{y'_{x}}
}_{\frac{dy}{du}}
\underbrace{
\dfrac{dx}{dy}
}_{\frac{du}{dx}}
$
Первый множитель преобразуем по формуле произвоной частного, второй - это величина, обратная производной
(она же производная обратной функции)

$
\dfrac{d}{dx}
\dfrac{1}{y'_{x}}
\underbrace{
\dfrac{dx}{dy}
}_{\frac{1}{y'_x}}
= -\dfrac{y''_{xx}}{(y'_{x})^{2}}
\dfrac{1}{y'_{x}}
=
- \dfrac{y''_{x}}{(y'_{x})^{3}}
$

$$
x''_{y} = - \dfrac{y''_{x}}{(y'_{x})^{3}} \label{diff1}
$$


\subsection{Дифференциалы высших порядков}

$ dy = f'(x)dx $

$ d^{2}y = d(dy) = d(f'(x)dx) = d(f'(x)) dx = (f''(x) dx) dx = f''(x) dx^{2} $

Дифференциалы n-го порядка определюятся рекуррентно.

$ d^{n}y = d(d^{n-1}y) $	

$ f^{(n)}(x) = \dfrac{d^{n}y}{dx^{n}} $

\subsection{Дифференцирование функции, заданной параметрически}


\subsubsection{Определение}

Пусть параметрические уравнения

$ x = \phi(t) $

$ y = \psi(t) $

задают плоскую кривую $ M(x,y) $

Например,

$ x = R \cos(t) $

$ y = R \sin(t) $

Задает окружность радиуса R

\subsubsection{Вычисление производной параметрической функции}

Найдем обратную функцию для одного из параметрических уравнений

$ x = \phi(t), t = g(x) $

$ y = \psi(t) = \psi( g(x) ) $

$ y'_{x} = y'_{t} t'_{x} = \dfrac{y'_{t}}{x'_{t}} $

Таким образом, производная сложной функции

$$ 
\dfrac{dy}{dx} = \dfrac{\psi'(t)}{\phi'(t)}, \phi'(t) \ne 0
$$

Возвращаясь к примеру с окружностью, производная этой параметрической функции будет 
$ \left( \dfrac{R \sin t}{R \cos t} \right)' =  -\ctg x $

\subsubsection{Вычисление производной высшего порядка для параметрической функции}

Опять найдем обратную для одной из функций

$ x = \phi(t), t = g(x) $

$ y = \psi(t) = \psi( g(x) ) $

Вычисление аналогично вычислению производных высших порядков обратной функции (TODO) $ \ref{diff1} $,
т.е. переходим к сложной функции и дифференциреем её как частное.

$ \dfrac{d^{2}y}{dx^{2}} = 
\underbrace{
\dfrac{d}{dx} 
\left( \dfrac{dy}{dx} \right)
}_{\frac{dy}{dx} = \frac{dy}{du}\frac{du}{dx}} =
\dfrac{d}{dt} 
\left( 
\dfrac{\psi'(t)}{\phi'(t)} 
\right)
\underbrace{
\dfrac{dt}{dx}
}_{\frac{1}{\phi'(t)}} =
\dfrac{ \psi''(t)\phi'(t)-\psi'(t)\phi''(t)  }{(\phi'(t))^{2}}
\dfrac{1}{\phi'(t)}
=
\dfrac{ \psi''(t)\phi'(t)-\psi'(t)\phi''(t)  }{(\phi'(t))^{3}}
$

Производная второго порядка для параметрических функций:

$$
\dfrac{d^{2}y}{dx^{2}}
=
\dfrac{ \psi''(t)\phi'(t)-\psi'(t)\phi''(t)  }{(\phi'(t))^{3}}
$$

Рекуррентная формула производной $n$-го порядка для параметрических функций:

$$
y^{(n)}_x = \dfrac{ (y^{(n-1)}_x)'_t }{x'_t}
$$


\subsubsection{Пример вычисления производной параметрической функции}

TODO : чего-то сомневаюсь я в этом примере

Функция задана параметрически:

$ x = a(t - \sin t) $

$ y = a(1 - \cos t) $

Найти первую и вторую производные

$ \dfrac{dy}{dx} = \dfrac{dy/dt}{dx/dt} = \dfrac{a(1 - \cos t) }{a(t - \sin t)}$
$ = \dfrac{ a\sin t }{a(1 - \cos t)} = \ctg \dfrac{t}{2} $

$ \dfrac{d^{2}y}{dx^{2}} = \dfrac{(\ctg \dfrac{t}{2})'}{ (a(t - \sin t))' } $
$ = - \dfrac{1}{\sin^{2} \frac{t}{2} } \dfrac{1}{2}  \dfrac{1}{a(1 - \cos t)}$
$ = - \dfrac{1}{4a \sin^{4} \frac{t}{2}} $


\subsection{Вектор-функция скалярного аргумента}

Вектор-функция — функция, значениями которой являются векторы в векторном пространстве $\mathbb V$ двух, трёх или более измерений.
Аргументами функции могут быть:

\begin{itemize}
\item
одна скалярная переменная -- тогда значения вектор-функции определяют в $\mathbb V$ некоторую кривую

\item
$m$ скалярных переменных -- тогда значения вектор-функции образуют в $\mathbb V$, вообще говоря, $m$-мерную

\item
векторная переменная -- в этом случае вектор-функцию обычно рассматривают как векторное поле на $\mathbb V$ поверхность
\end{itemize}

Пусть вектором $ r $ заданы точка, движущаяся по кривой:  $ r = r(t) $

Тогда можно определить её скорость $ v = v(t) $ и ускорение $ w = w(t) $

Как известно, вектор в пространстве задается тремя координатными ортами, $ i,j,k $

Тогда функция $ r = r(t) = a(t)i + b(t)j + c(t)k $

И задается этими тремя уравнениями параметрически:

$ x = a(t)i $

$ y = b(t)j $

$ z = c(t)k $

График такой функции называется \textbf{годографом}.


\subsubsection{Предел вектор-функции}


Определение

Пусть вектор-функция $ a(t) $ задана в окрестности некоторой точки.

Пределом вектор-функции называется постоянный вектор $ A = \lim\limits_{t \to t_{0}}  a(t)$, такой что:

$ \forall \epsilon > 0 \exists \delta > 0 : \forall t \ne t_{0} : |t-t_{0}| : |a(t) - A| < \epsilon $

Геометрически это означает стремление ветора $ a(t) $ к вектору $ A $, а их разности - к нулевому вектору.


\subsubsection{Непрерывная вектор-функция}

По аналогии с обычной функцией скалярного аргумета:

$ \lim\limits_{t \to t_{0}}  a(t) = a(t_{0}) $

\subsubsection{Производная вектор-функции}

Определим производную вектор-функции $\mathbf{r}(t)$ по параметру:

$\frac{d}{dt}\mathbf{r}(t)=\lim\limits_{h\to 0}\frac{\mathbf{r}(t+h) - \mathbf{r}(t)}{h}$.
Если производная в точке $t$ существует, вектор-функция называется дифференцируемой в этой точке. Координатными функциями для производной будут $x'(t),\ y'(t),\ z'(t)$:

$$
r'(t) = i x'(t) + j y'(t) + k z'(t) \label{vector_derivative}
$$

Формула выводится из определения производной ($ x'(t) = \dfrac{\Delta x(t)}{\Delta t}, $
$y'(t) = \dfrac{\Delta y(t)}{\Delta t}, \cdots~$)

Пример:

$ a(t) = i R\sin t + j R\cos t + k h t$

$ a'(t) = i R\cos t - j R\sin t + k h $

\subsubsection{Свойства производной вектор-функции}

$ \frac{d}{dt} (\mathbf{r_1}(t)+\mathbf{r_2}(t))=\frac{d\mathbf{r_1}(t)}{dt}+ \frac{d\mathbf{r_2}(t)}{dt}$ -- производная суммы есть сумма производных

$ \frac{d}{dt} (f(t)\mathbf{r}(t))=\frac{df(t)}{dt}\mathbf{r}(t) + f(t)\frac{d\mathbf{r}(t)}{dt} $ -- здесь $ f(t)$ дифференцируемая скалярная функция

$ \frac{d}{dt} (\mathbf{r_1}(t)\mathbf{r_2}(t))=\frac{d\mathbf{r_1}(t)}{dt}\mathbf{r_2}(t) + \mathbf{r_1}(t)\frac{d\mathbf{r_2}(t)}{dt} $ -- дифференцирование скалярного произведения (следствие - производная единичного вектора перпендикулярна ему, т.к. скалярное произведение равно $ 0 $)

$ \frac{d}{dt} [\mathbf{r_1}(t)\mathbf{r_2}(t)]=\left [\frac{d\mathbf{r_1}(t)}{dt}\mathbf{r_2}(t)\right ] + \left [\mathbf{r_1}(t) \frac{d\mathbf{r_2}(t)}{dt}\right] $ дифференцирование векторного произведения

$ \frac{d}{dt} (\mathbf{a}(t),\mathbf{b}(t),\mathbf{c}(t))=\left (\frac{d\mathbf{a}(t)}{dt},\mathbf{b}(t),\mathbf{c}(t)\right) + \left (\mathbf{a}(t),\frac{d\mathbf{b}(t)}{dt},\mathbf{c}(t)\right) + \left (\mathbf{a}(t), \mathbf{b}(t), \frac{d\mathbf{c}(t)}{dt}\right) $ -- дифференцирование смешанного произведения

