\section{Теоремы о среднем. Правило Лопиталя раскрытия неопределенностей}

\subsection{Теоремы о среднем. Пример 1}

$ f(x) = \dfrac{x^{2} - 2}{x^{4}} $

Имеет ли производная значение $ f'(x) = 0 $ на промежутке $ [-1,1] $

$ f(-1) = f(1) = -1 $

Однако теоремой Ролля мы воспользоваться не можем, т.к. функция не является непрерывной на промежутке $ [-1,1] $
(разрывна в точке $ 0 $)

\subsection{Теоремы о среднем. Пример 2}

Найти формулу Лагранжа для функции $ f(x) = x^{2} $ на промежутке $ [0,1] $

$ f'(x) = 2x $

$ f(1) - f(0) = f'(\xi)(1-0)$

$ \xi = \dfrac{1}{2} $

\subsection{Теоремы о среднем. Пример 3}

Доказать, что уравнение $ x^{3} + x + a = 0 $ имеет единственный корень

Очевидно, на каком-то промежутке $ [-x_0, x_0] $ по Т. Больцано-Коши найдется хотя бы одно нулевое значение.

Покажем, что оно единственное.

Предположим, что $ \exists x_1 \ne x_2 $, также являющееся корнем уравенения.

Рассмотрим отрезок $ [x_1, x_2] $. Тогда по Т. Ролля там найдется нулевое значение производной,
т.к. $ f(x_1) = f(x_2) = 0 $.

$ f'(x) = 3x^{2}+1 $ - очевидно, нигде не равно нулю. Пришли к противоречию, утверждение доказано.

\subsection{Раскрытие неопределенностей. Пример 1}

$ \lim\limits_{x \to 0} x \ctg 5x = \dfrac{x}{\tg 5x} = $

по правилу Лопиталя

$ = \dfrac{1}{5x/\cos 5x} = \dfrac{1}{5} $

\subsection{Раскрытие неопределенностей. Пример 2}

$ \lim\limits_{x \to 0} (\cos x)^{1/x} = e^{\lim\limits_{x \to 0} \frac{1}{x} \ln \cos x} $

$ \lim\limits_{x \to 0} \dfrac{\ln \cos x}{x} =  $

по правилу Лопиталя

$ = \lim\limits_{x \to 0} \dfrac{\cos x}{(\cos x)'} = \lim\limits_{x \to 0} -tg x = 0 $

$ \lim\limits_{x \to 0} (\cos x)^{1/x} = e^{\lim\limits_{x \to 0} \frac{1}{x} \ln \cos x} = e^{0} = 1$