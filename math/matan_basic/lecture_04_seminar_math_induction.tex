\section{Семинар: Множества. Метод математической индукции}

\subsection{Рациональные числа}

Представление периодической дроби в виде рациональной

Предположим, что дана периодическая десятичная дробь $x=0,(1998)$ с периодом 4. Заметим, что домножив её на $10^4 = 10000$, получим большую дробь $10000x=1998,(1998)$ с теми же цифрами после запятой. Отняв целую часть, получаем $10000x-1998=x \Rightarrow x=\frac{1998}{9999}=\frac{222}{1111}$

Т.е.  формула перевода в рациональную дробь

$$
\dfrac{m}{n} = a_0,a_1 a_2 \ldots a_m (b_1 \ldots b_k) 
= a_0,a_1 a_2 \ldots a_m + \dfrac{b_1 \ldots b_k}{ \underbrace{9 \ldots 9}_{k} }10^{-m}
$$

Например, $1,(2) = 1 + \dfrac{2}{9} 10^0$, или $3,00(3) = 3 + \dfrac{3}{9}10^{-2}$

\subsection{Иррациональные числа}

Доказать иррациональность числа ($ \sqrt{2}, \sqrt{3}, \ldots $)

Доказывается так: $\sqrt{3} = \frac{m}{n}, 3n^2 = m^2$. Значит, 3 - общий делитель чисел $m, n $, а это невозможно, т.к. дробь $ \frac{m}{n} $ несократимая.

\subsection{Метод математической индукции}

Индукцией называют переход от частных утверждений к общим. Напротив, переход от общих утверждений к частным называется дедукцией.

Пример частного утверждения: 254 делится на 2 без остатка.

Из этого частного утверждения можно сформулировать массу более общих утверждений, причем как истинных так и ложных. К примеру, более общее утверждение, что все целые числа, оканчивающиеся четверкой, делятся на 2 без остатка, является истинным, а утверждение, что все трехзначные числа делятся на 2 без остатка, является ложным.

\subsubsection{Пример 1}

Доказать
$1^2 + 2^2 + \ldots + n^2 = \dfrac{n(n+1)(2n+1)}{6}$

Докажем для 1: $1 = \dfrac{1 \cdot 3}{6}$

Докажем для $n+1$:

$1^2 + 2^2 + \ldots + n^2 + (n+1)^2 = \dfrac{n(n+1)(2n+1)}{6} + (n+1)^2 = $

$ = \dfrac{n(n+1)(2n+1)}{6} + \dfrac{6(n+1)^2}{6} = $

$= \dfrac{n(n+1)(2n+1) + 6(n+1)^2}{6} = $

$ = \dfrac{(n+1)( n(2n+1) + 6(n+1) )}{6} = $

считаем выражение в скобках

$ = \dfrac{(n+1)( (n+2)(2n+3) )}{6} = $

$ = \dfrac{(n+1)( ((n+1) + 1) (2(n+1) + 1) )}{6}$

Т.е. исходное выражение, где вместо $ n $  подставлено $ n+1 $

Доказано

\subsubsection{Пример 2. Неравенство Бернулли}

$(1+x)^n \ge 1 + nx; (x \ge -1, n \in N  ) $

Для $ n = 1, n = 2 $ - очевидно

Докажем для $n+1$:

$ (1+x)^{n+1} = (1+x)^{n}(1+x) > (1+nx)(1+x) $, т.к. для $ n $ предположение верно

$ (1+nx)(1+x) = 1 + nx + x + nx^{2} = 1+ x(n+1) + nx^{2} $, 

заметим, что первые 2 слагаемых  - правая часть неравенства Бернулли для $ n+1 $

А т.к. заведомо $ 1 + x(n+1) + nx^{2} > 1 + x(n+1) $, то утверждение доказано
