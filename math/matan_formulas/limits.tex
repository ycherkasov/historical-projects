\documentclass[12pt]{article}

% AMS
\usepackage{amsmath}
\DeclareMathOperator{\tg}{tg}
\DeclareMathOperator{\ctg}{ctg}

\usepackage{amsfonts}
\usepackage{amssymb}

% cyrillic support
\usepackage[T1]{fontenc}
\usepackage[utf8]{inputenc}
\usepackage[english,russian]{babel} 

% 
\setlength{\topmargin}{-1in}
\setlength{\headheight}{0.1in}
\setlength{\oddsidemargin} {-0.5in}
\setlength{\textwidth}{6in}
\setlength{\textheight}{8.9in}
\setlength{\marginparwidth}{0in}
\setlength{\parskip}{4pt}
\setlength{\parindent}{0pt}


\begin{document}

Basics: \\
$P(n)=a_1 x^n + a_2 x^{n-1} + \cdots + a_{n-1}x +  a_{n}$ \\
$P(m)=a_1 x^m + a_2 x^{m-1} + \cdots + a_{m-1}x +  a_{m}$ \\

$n < m: 
\lim_{x\to \infty} \frac{P(n)}{P(m)} = 0 $\\

$n = m: 
\lim_{x\to \infty} \frac{P(n)}{P(m)} = Const $\\

$n > m: 
\lim_{x\to \infty} \frac{P(n)}{P(m)} = \infty $\\

Patterns \\
1. $\lim_{x\to x_0} \sqrt{a} - \sqrt{b} = 
\frac{(\sqrt{a} - \sqrt{b})(\sqrt{a} + \sqrt{b})}{(\sqrt{a} + \sqrt{b})} =
\frac{a - b}{(\sqrt{a} + \sqrt{b})}  = \cdots $

2. $ \lim_{x\to x_0} \log_a{f(x)} =  \log_a {\lim_{x\to x_0} {f(x)} } =  $

3. $ \lim_{x\to x_0} e^{f(x)} =  e^ {\lim_{x\to x_0} {f(x)} }  $ \\

Remarkable limits \\
1. $ \lim_{x\to 0} \frac{\sin{x}}{x} = 0 (\lim_{x\to 0} \frac{\tan{x}}{x} = 0) $ \\
2. $ \lim_{x\to 0} (1 + 1/x)^x = e $ \\
3. $ \lim_{x\to 0} \frac{\ln{(1+x)}}{x} = 1 $ \\
4. $ \lim_{x\to 0} \frac{a^x + 1}{x} = \ln{a} $ \\
5. $ \lim_{x\to 0} \frac{(1 + x)^a - 1}{x} = a $ \\

Landau symbols \\
$f = o(g) \Leftrightarrow \lim_{n to 0} \frac{f(x)}{g(x)} = 0$ \\
e.g. $g(x) >> f(x)$ in $U_x$ \\

$f = O(g) \Leftrightarrow |f(n)| \le C|g(n)|q $ \\
e.g. $g(x) limited by f(x)$ in $U_x$ \\

$f ~ g \Leftrightarrow \lim_{n to 0} \frac{f(x)}{g(x)} = 1 $ \\
e.g. $g(x) and f(x)$ are equivalent \\

O-notation formulas \\
$o(f)+o(g) = o(f)$

$o(f) \cdot o(g) = o(f \cdot g)$

$o(o(f)) = o(f)$

$o(f)+O(f) = O(f)$

$o(f) \cdot O(f) = O(f \cdot g)$

$O(f) \cdot O(g) = O(f \cdot g)$

$O(o(f)) = o(f)$ \\\\

$ \alpha < \beta $ \\
$x^{\alpha} = o(x^{\beta}, x\to \infty ) $ \\
$x^{\beta} = o(x^{\alpha}, x\to 0 )$ \\

Examples: \\
$ f(x) = o( 1 ) $ \\
$ x = o( x^2 ), x\to \infty $ \\
$ x^2 = o( x ), x\to 0 $ \\

Taylor formula \\
$$
f(x) = \sum_{k = 0}^{n - 1} \frac{f^{(k)} (a)}{k!} (x - a)^k + R_n(x), x \in (a, b)
$$
$$
R_n(x) = \frac{f^{(n)}(a + \Theta(x - a))}{n!} 
$$

Taylor series \\
1. $e^x = 1 + x + \frac{x^2}{x!} + \frac{x^3}{3!} + \cdots + \frac{x^n}{n!} + o(x^n) $ \\
2. $ \sin{x} = x - \frac{x^3}{3!} + \frac{x^5}{5!} + \cdots + (-1)^{n-1}\frac{x^{2n-1}}{(2n-1)!} + o(x^{2n}) $ \\
3. $ x = x - \frac{x^2}{2!} + \frac{x^4}{4!} + \cdots + (-1)^n\frac{x^{2n}}{(2n)!} + o(x^{2n+1}) $ \\
4. $ (1+x)^n = 1 + mx + x^2\frac{m(m-1)}{2!} + x^n\frac{m(m-1)\cdots(m-n+1)}{n!} + o(x^n), x \in (-1,1) $ \\
5. $ \ln(x + 1) = x - \frac{x^2}{2} + \frac{x^3}{3} - \cdots + (-1)^{n-1}\frac{x^n}{n} + o(x^n), x \in (-1,1) $ \\

Limits and derivatives \\
1. $\lim_{h\to 0}{f(x+h)-f(x)\over h}=f'(x)$ \\
2. $\lim_{h\to0}\left(\frac{f(x+h)}{f(x)}\right)^\frac{1}{h}=\exp\left(\frac{f'(x)}{f(x)}\right)$ \\
3. $\lim_{h \to 0}{ \left({f(x(1+h))\over{f(x)}}\right)^{1\over{h}} }=\exp\left(\frac{x f'(x)}{f(x)}\right)$ \\

Notable special limits \\
$\lim_{x\to+\infty} \left(1+\frac{k}{x}\right)^{mx}=e^{mk} $ \\
$\lim_{x\to+\infty} \left(1-\frac{1}{x}\right)^x=\frac{1}{e} $ \\
$\lim_{x\to+\infty} \left(1+\frac{k}{x}\right)^x=e^k $ \\
$\lim_{n\to\infty} \frac{n}{\sqrt[n]{n!}}=e $ \\
$\lim_{n\to \infty }\, 2^{n} \underbrace{\sqrt{2-\sqrt{2+\sqrt{2+\text{...} +\sqrt{2}}}}}_n= \pi $ \\

Logarithmic and exponential functions: \\
$\mbox{For } a > 1: \, $ \\
$\lim_{x \to 0^+} \log_a x = -\infty$ \\
$\lim_{x \to \infty} \log_a x = \infty$ \\
$\lim_{x \to -\infty} a^x = 0 $ \\
$\mbox{If } a < 1: \, $ \\
$\lim_{x \to -\infty} a^x = \infty $ \\

Trigonometric functions \\
$\lim_{x \to a} \sin x = \sin a $ \\
$\lim_{x \to a} \cos x = \cos a $ \\
$\lim_{x \to 0} \frac{\sin x}{x} = 1 $ \\
$\lim_{x \to 0} \frac{1-\cos x}{x} = 0 $ \\
$\lim_{x \to 0} \frac{1-\cos x}{x^2} = \frac{1}{2} $ \\
$\lim_{x \to n^\pm} \tan \left(\pi x + \frac{\pi}{2}\right) = \mp\infty \qquad \text{for any integer } n $ \\

\end{document}