\section{Семинар: Числовая последовательность и ее предел}

\subsection{Доказательство существования предела по определению}

Докажем, что $ a = \lim\limits_{n \to \infty} \dfrac{1}{q^{n}} = 0$

по определению предела числовой последовательности

$\forall \varepsilon > 0 \exists N: \forall n: n > N \Rightarrow |x_{n} - a| < \varepsilon$

Т.е. в нашем случае $  |x_{n} - a| = |x_{n} - 0| = | \dfrac{1}{q^{n}} | = \dfrac{1}{q^{n}} $

Прибавим и отнимем единицу 

$ \dfrac{1}{q^{n}} = 1 + \left( \dfrac{1}{q} - 1 \right)  $

По неравенству Бернулли

$ 1 + ( \dfrac{1}{q} - 1 )^{n} > 1 + n ( \dfrac{1}{q} - 1 ) > n ( \dfrac{1}{q} - 1 ) $

$ q^{n} <  \dfrac{1}{n ( \dfrac{1}{q} - 1 )}  $

$ q^{n} <  \dfrac{q}{n - nq} = \dfrac{q}{n(1 - q)} $

Т.е. $ \forall n > \dfrac{q}{n(1 - q)} : | \dfrac{1}{q^{n}} | < \varepsilon $

Т.е. 0 - является пределом

Также, если $ n < 1 $, данное утверждение неверно, и предел 
$ \lim\limits_{n \to \infty} \dfrac{1}{q^{n}} = \infty $

\subsection{Найти число N, начиная с которого последовательность сходится}

$ \lim\limits_{n \to \infty} \dfrac{9n + 7}{2n - 3} = \dfrac{9}{2}, \varepsilon = 10^{-2}$

найдем N, начиная с которого $  |x_{n} - a| < \varepsilon $

$ |x_{n} - a| = \left|  \dfrac{9n + 7}{2n - 3} - \dfrac{9}{2} \right| = $
$ \left|  \dfrac{41}{2(2n - 3)} \right| < \varepsilon $

Разрешим неравенство относительно n

$ \left|  \dfrac{41}{4n - 6} \right| < \varepsilon $

$ \dfrac{41}{4n} - \dfrac{41}{6}  < \varepsilon $

$ \dfrac{41}{4n} < \varepsilon + \dfrac{41}{6}$

$ n < \dfrac{4}{41}\varepsilon + \dfrac{4}{41}\dfrac{41}{6}$

Возведем в $ -1 $ степень

$ n > \dfrac{41}{4}\varepsilon + \dfrac{3}{2}$

Если $ \varepsilon = 10^{-2} $, то $ N(\varepsilon) = \dfrac{4100}{4} + \dfrac{3}{2} = 1028.5 $

$ N = 1029 $

\subsection{Пределы многочленов (общий подход)}

Пусть $ P(m), P(n) $ - многочлены $ m $ и $ n $ степени соотетственно

$P(n)=a_1 x^n + a_2 x^{n-1} + \cdots + a_{n-1}x +  a_{n}$ \\

$P(m)=a_1 x^m + a_2 x^{m-1} + \cdots + a_{m-1}x +  a_{m}$ \\

Тогда

$n < m: \lim\limits_{x\to \infty} \dfrac{P(n)}{P(m)} = 0 $

$n = m: \lim\limits_{x\to \infty} \dfrac{P(n)}{P(m)} = С $

$n > m: \lim\limits_{x\to \infty} \dfrac{P(n)}{P(m)} = \infty $

\subsection{Пределы с выражениями под знаком радикала (общий подход)}

$\lim\limits_{x\to x_0} \sqrt{a} - \sqrt{b} = 
\dfrac{(\sqrt{a} - \sqrt{b})(\sqrt{a} + \sqrt{b})}{(\sqrt{a} + \sqrt{b})} =
\dfrac{a - b}{(\sqrt{a} + \sqrt{b})}  = \cdots $

\subsection{Пределы с выражениями под знаком логарифма(экспоненты) (общий подход)}

$ \lim\limits_{x\to x_0} \log_a{f(x)} =  \log_a {\lim\limits_{x\to x_0} {f(x)} }  $

$ \lim\limits_{x\to x_0} e^{f(x)} =  e^ {\lim\limits_{x\to x_0} {f(x)} }  $

\subsection{Пределы с заданной рядом формулой общего члена (Примеры)}

\subsubsection{Пример. Найти предел}

Пусть общий член задан формулой

$ x_{n} = \dfrac{1}{1 \cdot 2} + \dfrac{1}{2 \cdot 3} + \ldots + \dfrac{1}{n (n+1)}$

Общий подход - разложим каждый член так, чтобы их можно было сократить

$ x_{n} =  \left( 1 - \dfrac{1}{2} \right) +
\left(  \dfrac{1}{2} - \dfrac{1}{3} \right) +
\left(  \dfrac{1}{3} - \dfrac{1}{4} \right) +
\ldots +
\left(  \dfrac{1}{n} - \dfrac{1}{n+1} \right) =
 $
 
 Сокращаем все слагаемые
 
 $ = 1 - \dfrac{1}{n+1} $
 
 Очевидно $ \lim\limits_{n \to \infty} x_{n} = 1 - \dfrac{1}{n+1} = 1 $

\subsubsection{Пример. Найти предел}

$ x_{n} = \dfrac{2^{n}}{2!} $

найдем путем оценки n-го члена и предельного перехода в неравенствах

$ \dfrac{2^{n}}{2!} = \dfrac{2}{1}\dfrac{2}{2}\dfrac{2}{2} \ldots \dfrac{2}{n} $

Произведем оценку сверху, заменив все члены начиная с 3-го на $ \dfrac{2}{3} $

$ \dfrac{2}{1}\dfrac{2}{2}\dfrac{2}{2} \ldots \dfrac{2}{n} <
\dfrac{2}{1}\dfrac{2}{2}\dfrac{2}{3}\dfrac{2}{3} \ldots \dfrac{2}{3} = 
\dfrac{9}{2} \left( \dfrac{2}{3} \right)^{n}
$

Очевидно

$ 0 < \dfrac{2^{n}}{2!} < \dfrac{9}{2} \lim\limits_{n \to \infty} \left( \dfrac{2}{3} \right)^{n} $

Правый предел равен 0, следовательно по "лемме о милиционерах"

$ \lim\limits_{n \to \infty} x_{n} = \dfrac{2^{n}}{2!} = 0 $

\subsubsection{Пример. Найти предел}


Пусть общий член задан формулой

$ x_{n} = \dfrac{1 + a + a^{2} + a^{3} + \ldots + a^{n}}{1 + b + b^{2} + b^{3} + \ldots + b^{n}}$

$ |a| < 1, |b| < 1 $

По формуле о сумме бесконечно убывающей геометрической прогрессии

$$
S = \dfrac{b_{1}}{1 - q}
$$

$ x_{n} = \dfrac{1 + a + a^{2} + a^{3} + \ldots + a^{n}}{1 + b + b^{2} + b^{3} + \ldots + b^{n}}
= \dfrac{1 - b}{1 - a}
$

Заметим, что получившееся выражение не зависит от $ n $, поэтому

$ \lim\limits_{n \to \infty}  \dfrac{1 - b}{1 - a} =  \dfrac{1 - b}{1 - a} $


\subsubsection{Пример. Найти предел}

Пусть общий член задан формулой

$ x_{n} = \dfrac{1^{2}}{n^{3}} + 
\dfrac{2^{2}}{n^{3}} + 
\ldots +
\dfrac{(n-1)^{2}}{n^{3}}
 $

Вынесем $ \frac{1}{n^{3}} $ за скобки

$ x_{n} = \dfrac{1}{n^{3}} (1^{2} + 2^{2} + \ldots + (n-1)^{2}) $

Воспользуемся формулой для суммы квадратов

$ (1^{2} + 2^{2} + \ldots + (n-1)^{2}) = 
\dfrac{(n - 1)((n - 1) + 1)(2(n-1) + 1)}{6} = 
\dfrac{n(n - 1)(2n -1)}{6}
$

$ \lim\limits_{n \to \infty} x_{n} = \dfrac{1}{n^{3}} \dfrac{n(n - 1)(2n -1)}{6}
= \dfrac{2n^{3}-3n^{2}-n}{6n^{3}} = \dfrac{1}{3}
$


\subsubsection{Пример. Найти предел}

Пусть общий член задан формулой

$ x_{n} = \sqrt{2} \sqrt[4]{2} \sqrt[8]{2} \ldots \sqrt[2^{n}]{2} $

Немного преобразуем

$ \sqrt{2} \sqrt[4]{2} \sqrt[8]{2} \ldots \sqrt[2^{n}]{2} =
2^{\frac{1}{2} + \frac{1}{4} + \frac{1}{8} + \ldots + + \frac{1}{2^{n}}}
$

Показатель - убывающая геометрическая прогрессия

По формуле о сумме бесконечно убывающей геометрической прогрессии

$S = \dfrac{b_{1}}{1 - q} = \dfrac{1/2}{1 - 1/2} = 1$

$  \lim\limits_{n \to \infty} x_{n} = 
\sqrt{2} \sqrt[4]{2} \sqrt[8]{2} \ldots \sqrt[2^{n}]{2} = 
2^{1} = 2 $


\subsection{Доказательство сходимости или расходимости последовательности по критерию Коши (Примеры)}

\subsubsection{Пример. Доказать сходимость}

$ x_{n} = 1 +
\dfrac{1}{2^{2}} + 
\dfrac{1}{3^{2}} + 
\ldots +
\dfrac{1}{n^{2}}
$

Согласно критерию Коши

$ \forall \varepsilon > 0 \exists N: \forall n,m > N \Rightarrow |x_{n} - x_{m}| < \varepsilon $

Пусть n-й член последовательности $ x_{n} $, 
m-й член последовательности 

$ x_{n+p} = 1 + \dfrac{1}{2^{2}} + 
\dfrac{1}{3^{2}} + 
\ldots +
\dfrac{1}{(n+p)^{2}} $

Тогда по критерию Коши

$ |x_{n+p} - x_{n}| = 
\dfrac{1}{(n+1)^{2}} +
\dfrac{1}{(n+2)^{2}} +
\ldots +
\dfrac{1}{(n+p)^{2}} $

Оценим выражение сверху, заменив каждый делитель на меньший (один из множителей минус 1)

$ 
\dfrac{1}{(n+1)^{2}} +
\dfrac{1}{(n+2)^{2}} +
\ldots +
\dfrac{1}{(n+p)^{2}}
<
\dfrac{1}{n(n+1)} +
\dfrac{1}{(n+2)(n+1)} +
\ldots +
\dfrac{1}{(n+p)(n+p-1)}
$

Воспользуемся приемом сокращения, разложим каждое слагаемое на множители 
($ \frac{1}{AB} = \frac{1}{A} + \frac{1}{B} $)

$
\dfrac{1}{n(n+1)} +
\dfrac{1}{(n+2)(n+1)} +
\ldots +
\dfrac{1}{(n+p)(n+p-1)} =
$

$
=
\left( \dfrac{1}{n} - \dfrac{1}{n+1} \right) +
\left( \dfrac{1}{n+1} - \dfrac{1}{n+2} \right) +
\ldots +
\left( \dfrac{1}{n+p-1} - \dfrac{1}{n+p} \right)
=
\dfrac{1}{n} - \dfrac{1}{n+p} < \dfrac{1}{n} < \varepsilon
$

Таким образом, мы нашли $ N $, зависящую от $ \varepsilon $, а именно

$ N(\varepsilon) > \dfrac{1}{\varepsilon} $

, что выполняется критерий Коши

\subsubsection{Пример. Доказать рассходимость}

$ x_{n} = 1 +
\dfrac{1}{2} + 
\dfrac{1}{3} + 
\ldots +
\dfrac{1}{n}
$

Сформируем логически обратное условие заменой кванторов и знака в нервенстве

Прямое:

$ \forall \varepsilon > 0 \exists N: \forall n,m > N \Rightarrow |x_{n} - x_{m}| < \varepsilon $

Обратное:

$ \exists \varepsilon > 0 \forall N: \exists n,m > N \Rightarrow |x_{n} - x_{m}| > \varepsilon $

Далее - как в предыдущем примере