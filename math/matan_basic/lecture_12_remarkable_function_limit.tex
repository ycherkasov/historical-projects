\section{Семинар: Предел функции. Замечательные пределы}

\subsection{Обобщение 2го ЗП}

При решении многих пределов полезно использовать соледующий паттерн:

Если $ \exists  \alpha(x), \beta(x) $ такие, что 

$ \lim\limits_{x \to 0 } \alpha(x) = 0 $

$ \lim\limits_{x \to 0 } \beta(x) = 0 $

$ \lim\limits_{x \to a } \dfrac{\alpha(x)}{\beta(x)} = A $

$$
\lim\limits_{x \to a } \left( 1 + \alpha(x)  \right )^{\frac{1}{\beta(x)}} = e^{A}
$$

\subsection{Пример 1}

